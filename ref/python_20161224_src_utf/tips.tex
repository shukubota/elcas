\section{Pythonプログラミングのヒント}\label{sec: Pythonプログラミングのヒント}


\subsection{MATLABデータの読み込み}
	 PythonではMatlabのデータも扱うことができる.ここではその方法を記す.例えば,次のような構造体sが保存されたバージョンが7以前のMatlabファイル`test.mat'を考える.
	\begin{itemize}
		\item s.label =[`node1',`node2'] : 2$\times$5 char
		\item s.samplingrate = 0.05
		\item s.DATA : 2$\times$5$\times$2 double
			\begin{eqnarray*}
				{\rm s.DATA}(:,:,1)= 
					\begin{array}{ccccc}
						1 & 2 & 3 & 4 & 5\\
						6 & 7 & 8 & 9 & 10\\
					\end{array}, 
				{\rm s.DATA}(:,:,2)=
					\begin{array}{ccccc}
						11 & 12 & 13 & 14 & 15\\		
					16 & 17 & 18 & 19 & 20\\
					\end{array}
			\end{eqnarray*}
		\item s.average = [3,8;13,18] : 2$\times$2 double
	\end{itemize}
	このようなMatlabファイルを読み込むにはscipy.ioにあるloadmatを使えばよい.
\begin{lstlisting}[style=python]
d=scipy.io.loadmat(`$ファイル名')
\end{lstlisting}
	と読み込んだ変数dに対して,d[`\$変数名'][0,0][`\$フィールド名']とすれば構造体に含まれているフィールドの値を取得することができる.実際にtest.matを用意した上でコード\ref{code: loadmatのサンプル}を実行すると上記で記した構造体sが読み込まれていることを確認できる.
	\lstinputlisting[%
	style=python,%
	label={code: loadmatのサンプル},%
	caption={loadmatを用いて構造体sを読み込むコード.各フィールドの値を表示する.}]{code/loadmat.py}
	 他方でMatlabファイルのバージョンが7.3であれば,loadmatでファイルを読み込むことができない.この場合はパッケージh5pyを用いることになる.このh5pyはデフォルトではpythonには導入されていないのでh5pyをインストールする.	\begin{lstlisting}[style=cmdline]
> conda install h5py
\end{lstlisting}
	そしてh5pyにあるFileを用いて,
\begin{lstlisting}[style=python]
d=h5py.File(`$ファイル名','r')
\end{lstlisting}
	とすればよい.このようにして読み込んだdに対して,d[`\$変数名/\$フィールド名']のようにすれば構造体に含まれているフィールドの値を取得することができる.ただし,loadmatのときとは異なり,フィールドの値取得後の文字列の形式・次元の向きには注意が必要である.実際にバージョンが7.3であるtest.matを用意した上でコード\ref{code: h5pyのサンプル}を実行すると上記で記した構造体sが読み込まれていることを確認できる.これらのMatlabファイルの読み込みに関する詳細としてはリンク集にあるScipyのドキュメントおよびh5pyのドキュメントを参考されたい.
	\lstinputlisting[%
	style=python,%
	label={code: h5pyのサンプル},%
	caption={h5pyを用いて構造体sを読み込むコード.各フィールドの値を表示する.}]{code/h5py.py}
\subsection{図のファイル出力}
	図のファイル出力はmatplotlib.pyplotにあるsavefigを使えばよい.使い方としては,コード1におけるplt.plot()の部分を
\begin{lstlisting}[style=python]
plt.savefig('$ファイル名')
\end{lstlisting}
	とすればよい.出力画像のフォーマットとしては
	\begin{center}
		eps, pdf, pgf, png, ps, raw, rgba, svg, svgz
	\end{center}
	などが可能である.
\subsection{リンク集}
	\begin{itemize}
		\item PythonJapan(\url{http://www.python.jp/}):Pythonの日本語ドキュメント
		\item SciPy.org(\url{https://docs.scipy.org/doc/}):NumpyとScipyのドキュメント
		\item Matplotlib(\url{http://matplotlib.org/}):Matplotlibのドキュメント
		\item HDF5 for Python(\url{http://docs.h5py.org/}):h5pyのドキュメント
	\end{itemize}


